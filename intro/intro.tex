%%%%%%%%%%%%%%%%%%%%%%%%%%%%%%%%%%%%%%%%%
% Beamer Presentation
% LaTeX Template
% Version 1.0 (10/11/12)
%
% This template has been downloaded from:
% http://www.LaTeXTemplates.com
%
% License:
% CC BY-NC-SA 3.0 (http://creativecommons.org/licenses/by-nc-sa/3.0/)
%
% Modified by Jeremie Gillet in November 2015 to make an OIST Skill Pill template
%
%%%%%%%%%%%%%%%%%%%%%%%%%%%%%%%%%%%%%%%%%

%----------------------------------------------------------------------------------------
%	PACKAGES AND THEMES
%----------------------------------------------------------------------------------------

\documentclass{beamer}

\mode<presentation> {

\usetheme{}

\definecolor{OISTcolor}{rgb}{0.65,0.16,0.16}
\usecolortheme[named=OISTcolor]{structure}

%\setbeamertemplate{footline} % To remove the footer line in all slides uncomment this line
%\setbeamertemplate{footline}[page number] % To replace the footer line in all slides with a simple slide count uncomment this line

\setbeamertemplate{navigation symbols}{} % To remove the navigation symbols from the bottom of all slides uncomment this line
}

% Setting frametitle to be std
\setbeamercolor{frametitle}{bg=white, fg=black}
\setbeamertemplate{frametitle}[default][center]

%\usepackage{CJKutf8}
%\newcommand{\cntext}[1]{\begin{CJK}{UTF8}{gbsn}#1\end{CJK}}
\usepackage{amsmath,amsthm, amssymb, latexsym, mathptmx, mathtools}
\usepackage{graphicx} % Allows including images
%\usepackage{booktabs} % Allows the use of \toprule, \midrule and \bottomrule in tables
\usepackage{textpos} % Use for positioning the Skill Pill logo

% For code displays
\usepackage{listings}
\usepackage{color}
\usepackage{braket}
\usepackage{multimedia}

\setbeamertemplate{itemize items}[default]
\setbeamertemplate{enumerate items}[default]

\definecolor{dkgreen}{rgb}{0,0.6,0}
\definecolor{gray}{rgb}{0.5,0.5,0.5}
\definecolor{mauve}{rgb}{0.58,0,0.82}

\lstset{frame=tb,
  language=python,
  aboveskip=3mm,
  belowskip=3mm,
  showstringspaces=false,
  columns=flexible,
  basicstyle={\small\ttfamily},
  numbers=none,
  numberstyle=\tiny\color{gray},
  keywordstyle=\color{blue},
  commentstyle=\color{dkgreen},
  stringstyle=\color{mauve},
  breaklines=true,
  breakatwhitespace=true,
  tabsize=3
}



%----------------------------------------------------------------------------------------
%	TITLE PAGE
%----------------------------------------------------------------------------------------

\title[Presentation Skills]{Intro to GPU Computing} 
\author{\textbf{J. Schloss}} % Your name
\institute[OIST] % Your institution as it will appear on the bottom of every slide, may be shorthand to save space
{
Okinawa Institute of Science and Technology Graduate University\\ % Your institution for the title page
Quantum Systems Unit\\
\textit{james.schloss@oist.jp} % Your email address
}
\date{\today} % Date, can be changed to a custom date

\begin{document}

\usebackgroundtemplate{%
  \includegraphics[width=\paperwidth,height=\paperheight]{OISTBG}} 
\begin{frame}
\centering
    \titlepage % Print the title page as the first slide
\end{frame}

\usebackgroundtemplate{
    \includegraphics[width=\paperwidth,height=\paperheight]{PPTBG}}

\setbeamertemplate{background}{} % No background logo after title frame

\begin{frame}
\frametitle{Why use GPU computing?}

General Purpose Graphical Processing Unit (GPGPU) computing is a great step towards HPC computing on consumer hardware. It works best with programs that are:
\begin{itemize}
\item Data parallel (can act independently on different elements)
\item Throughput intensive (There are a lot of elements)
\end{itemize}

\begin{figure}
\begin{center}
\includegraphics[width=0.4\textwidth]{GPUE_BENCHMARKS.jpg}
\end{center}
\caption{\url{http://peterwittek.com/gpe-comparison.html}}
\end{figure}
\end{frame}

\begin{frame}
\frametitle{GPU vs CPU}
GPU Computing can do many things much faster than the CPU; however, there are a few drawbacks:
\begin{center}
\begin{tabular}{c | c | c}
& CPU & GPU \\
\hline
Memory & unlimited & ~12GB \\
& & \\
Parallelization & afterthought & natural \\
& OpenMP, MPI & CUDA, OpenCL...\\

\end{tabular}
\end{center}
\end{frame}

\begin{frame}
\frametitle{Parallelization}
\end{frame}

\begin{frame}
\frametitle{Examples}
\end{frame}

\begin{frame}
\frametitle{Hardware}
There are two major vendors for GPU's: 
\begin{description}
\item[AMD] ``Open'' computing
\item[nVidia] ``Industry standard'' for GPU computing. 
\end{description}

For the most part, these follow trends you hear about in gaming: nVidia trail-blazes and AMD keeps up; however, this is not necessarily true with recent cards.
\end{frame}

\begin{frame}
\frametitle{CUDA}

CUDA (once Compute Unified Device Architecture) is the standard programming language to use for GPGPU computing and boasts speed and performance; however, it only works on nVidia cards.
\end{frame}

\begin{frame}
\frametitle{CUDAnative.jl}
This is a julia implementation of CUDA and will be developed further in the future. It is much easier to use than CUDA and works well with most Julia code.
\end{frame}

\begin{frame}
\frametitle{OpenCL}

\begin{itemize}
\item OpenCL follows similar notation to OpenGL. In OpenGL, shaders are read in as strings, but in OpenCL, kernels are read in as strings
\end{itemize}
\end{frame}

\begin{frame}[fragile]
\frametitle{Accessing GPU's at OIST}

After logging on to Sango or Tombo, you can check which GPU's are taken with 

\begin{lstlisting}
squeue -p gpu
\end{lstlisting}

and can access a GPU node (interactively) with

\begin{lstlisting}
srun ... 
\end{lstlisting}

Once you have access, you can check each GPU with \texttt{nvidia-smi}
\end{frame}

\end{document} 
